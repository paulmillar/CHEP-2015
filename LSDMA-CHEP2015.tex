\documentclass[a4paper]{jpconf}

\bibliographystyle{iopart-num}
\usepackage{citesort}
\usepackage{graphicx}
\usepackage{caption}
%\usepackage{subcaption}

\def\tilde{\raise.17ex\hbox{$\scriptstyle\sim$}}

\begin{document}
\title{Unlocking data: federated identity with LSDMA and dCache}

\author{AP Millar$^1$, G~Behrmann$^2$, C~Bernardt$^1$, P~Fuhrmann$^1$,
  M~Hardt$^3$, A~Hayrapetyan$^3$, D~Litvintsev$^4$, T~Mkrtchyan$^1$,
  A~Rossi$^4$, K~Schwank$^1$}

\address{$^1$ IT Dept., DESY, Notkestrasse 85, Hamburg, Germany}
\address{$^2$ NORDUnet, Copenhagen, Denmark}
\address{$^3$ KIT}
\address{$^4$ Fermilab, Chicago, IL, USA}

\ead{paul.millar@desy.de}

\begin{abstract}
X.509, the dominant identity system from grid computing, has proved
unpopular for many user communties. More popular alternatives
generally assume the user is interacting via their web-browser. Such
alternatives allow a user to authenticate with many services with the
same credentials (username and password). They also allow users from
different organisations form collaborations quickly and simply.

Scientists generally require that their custom analysis software has
direct access to the data. Such direct access is not currently
supported by alternatives to X.509, as they require the use of a
web-browser.

Various approaches to solve this issue are being investigated as part
of the Large Scale Data Management and Analysis (LSDMA) project, a
German funded national R\&D project. These involve dynamic credential
translation (creating an X.509 credential) to allow backwards
compatibility in addition to direct SAML- and OpenID Connect-based
authentication.

We present a summary of the current state of art and the current
status of the federated identity work funded by the LSDMA project
along with the future road map.
\end{abstract}

\section{Introduction}

With the advent of the Large Hadron Collider (LHC) facility at CERN
and the corresponding world-wide grid infrastructure, the World-wide
LHC Computational Grid (WLCG), deploying and managing data that is
spread across the globe became a reality.  The community of scientists
that use wish to analyse this data is equally distributed, requiring a
common, decentralised authentication mechanism.

The adopted solution was to adopt X.509-based authentication, with
users issued with an X.509 certificate by their local certificate
authority (CA), which typically operate nationally.  These CAs are
trusted by the sites in WLCG providing resources through the CA's
membership of a global trust federation (the IGTF).  By trusting the
CAs, the sites can trust that the X.509 certificates they issue
identify the users.

The adoption of X.509 client certificates led to a world-wide,
decentralised network of resources for the analysis and storage of LHC
data.  This model has spread, as the technology brought into
production to support LHC use-cases is adapted to support other
communties.  However, X.509 client certificates remain a burden for
users: it is neither a technology they are used to, nor something that
existing software provides a good user experience.  The extent to
which this affects scientists is not to be understated: it is only the
larger user communites that have adopted X.509 client certificates as
their authentication mechanism as they have the resources to train
their users in a new authentication mechanism.

Given the difficulties faced with X.509-based authentication, several
projects have looked into providing a decentralised (or ``federated'')
authentication mechanisms as alternatives.  Perhaps the most widely
deployed currently is the SAML-based approach.

\section{Trust}

Users new to grid computing often experience problems in acquiring an
X.509 certificate, as the CA must vet the users identity.  For new
communities, it is harder as they often have no access to an
IGTF-accredited CA and establishing a new CA is prohibitively
expensive.

However, often this overhead is unnecessary as the user's home
institute has already vetted the user with at least the same level of
scrutiny as IGTF-membership requires.  Because of this, several
projects have appeared that allow the user to obtain their certificate
by authenticating against their home
institute\cite{fed-login-teragrid}.  While this is an improvement, it
still requires the use of X.509 certificates.  This is problematic as
most software does not make handling X.509 easy for the user.


\section{Direct access problem}

\section{LSDMA}

LSDMA\cite{lsdma} (Large Scale Data Management and Analysis) is a
project funded by the German government to link research of the
Helmholtz Association of research centres in Germany with community
specific Data Life Cycle Laboratories (DLCL). The DLCLs work in close
cooperation with scientists and they process, manage and analyse data
during its whole life cycle. The joint research and development
activities in the DLCLs lead to community-specific tools and
mechanisms. The DLCLs are complemented with a Data Services
Integration Team (DSIT). This provides generic technologies and
infrastructures for multi-community use, based on research and
development in the areas of data management, data access and security,
storage technologies and data preservation.

dCache, as major partner within the LSDMA DSIT team, is working on
solving these authentication problems through authenticating via SAML.
This is non-trivial as almost all SAML usage has been for web-driven
activity while most data transfers are not initiated by a web-browser.

Within DSIT, the plan is to establish a federation of Identity
Providers (IdPs) for Germany: LSDMA-AAI.  This federation is similar
to the existing DFN federation (DFN-AAI)\cite{dfn-aai-website} and we
anticipate LSDMA-AAI becoming part of DFN-AAI in the future.  The
initial work will be on providing web-portals that allow a user to
authenticate via SAML.  An X.509 credential is built from that
information, which is used when authenticating with data servers.
This follows existing work in this
direction\cite{groeper2007architecture}.

Group-membership is a common concern across many projects as often it
is desired to authorise some operation based on that user's group
membership.  A common group-membership service will be run within
Germany to allow services to discover group membership of a user.

Later, support in dCache for certain transfer protocols (e.g., FTP and
NFS) will be updated to allow direct SAML-based authentication.  This
will allow clients to access their data directly without a web-portal
using their home institute credentials in a secure fashion.


\section{Work undertaken}

Tunnelling SAML assertion through ssh password field.

Agreement with DFN-AAI to use their 

\section{Future work}

\section{Conclusion}


---

X.509, the dominant identity system from grid computing, has proved
unpopular for many user communties. More popular alternatives
generally assume the user is interacting via their web-browser. Such
alternatives allow a user to authenticate with many services with the
same credentials (username and password). They also allow users from
different organisations form collaborations quickly and simply.

Scientists generally require that their custom analysis software has
direct access to the data. Such direct access is not currently
supported by alternatives to X.509, as they require the use of a
web-browser.

Various approaches to solve this issue are being investigated as part
of the Large Scale Data Management and Analysis (LSDMA) project, a
German funded national R\&D project. These involve dynamic credential
translation (creating an X.509 credential) to allow backwards
compatibility in addition to direct SAML- and OpenID Connect-based
authentication.

We present a summary of the current state of art and the current
status of the federated identity work funded by the LSDMA project
along with the future road map.

\section{Conclusion}

\ack

Acknowledge LSDMA

\section*{References}
\bibliography{LSDMA-CHEP2015}

\end{document}
