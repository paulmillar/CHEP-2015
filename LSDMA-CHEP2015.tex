\documentclass[a4paper]{jpconf}

\bibliographystyle{iopart-num}
\usepackage{citesort}
\usepackage{graphicx}
\usepackage{caption}
%\usepackage{subcaption}

\def\tilde{\raise.17ex\hbox{$\scriptstyle\sim$}}

\begin{document}
\title{Unlocking data: federated identity with LSDMA and dCache}

\author{AP Millar$^1$, G~Behrmann$^2$, C~Bernardt$^1$, P~Fuhrmann$^1$,
  M~Hardt$^3$, A~Hayrapetyan$^3$, D~Litvintsev$^4$, T~Mkrtchyan$^1$,
  A~Rossi$^4$, K~Schwank$^1$}

\address{$^1$ IT Dept., DESY, Notkestrasse 85, Hamburg, Germany}
\address{$^2$ NORDUnet, Copenhagen, Denmark}
\address{$^3$ KIT}
\address{$^4$ Fermilab, Chicago, IL, USA}

\ead{paul.millar@desy.de}

\begin{abstract}
X.509, the dominant identity system from grid computing, has proved
unpopular for many user communties. More popular alternatives
generally assume the user is interacting via their web-browser. Such
alternatives allow a user to authenticate with many services with the
same credentials (username and password). They also allow users from
different organisations form collaborations quickly and simply.

Scientists generally require that their custom analysis software has
direct access to the data. Such direct access is not currently
supported by alternatives to X.509, as they require the use of a
web-browser.

Various approaches to solve this issue are being investigated as part
of the Large Scale Data Management and Analysis (LSDMA) project, a
German funded national R\&D project. These involve dynamic credential
translation (creating an X.509 credential) to allow backwards
compatibility in addition to direct SAML- and OpenID Connect-based
authentication.

We present a summary of the current state of art and the current
status of the federated identity work funded by the LSDMA project
along with the future road map.
\end{abstract}

\section{Introduction}

With the advent of the Large Hadron Collider (LHC) facility at CERN
and the corresponding world-wide grid infrastructure, the World-wide
LHC Computational Grid (WLCG), deploying and managing data that is
spread across the globe became a reality.  The community of scientists
that use wish to analyse this data is equally distributed, requiring a
common, decentralised authentication mechanism.

The adopted solution was to adopt X.509-based authentication, with
users issued with an X.509 certificate by their local certificate
authority (CA), which typically operate nationally.  These CAs are
trusted by the sites in WLCG providing resources through the CA's
membership of a global trust federation (the IGTF).  By trusting the
CAs, the sites can trust that the X.509 certificates they issue
identify the users.

The adoption of X.509 client certificates led to a world-wide,
decentralised network of resources for the analysis and storage of LHC
data.  This model has spread, as the technology brought into
production to support LHC use-cases is adapted to support other
communties.  However, X.509 client certificates remain a burden for
users: it is neither a technology they are used to, nor something that
existing software provides a good user experience.  The extent to
which this affects scientists is not to be understated: it is only the
larger user communites that have adopted X.509 client certificates as
their authentication mechanism as they have the resources to train
their users in a new authentication mechanism.

Given the difficulties faced with X.509-based authentication, several
projects have looked into providing a decentralised (or ``federated'')
authentication mechanisms as alternatives.  Perhaps the most widely
deployed currently is the SAML-based approach.

\section{Conclusion}

\ack

Acknowledge LSDMA

\section*{References}
\bibliography{LSDMA-CHEP2015}

\end{document}
