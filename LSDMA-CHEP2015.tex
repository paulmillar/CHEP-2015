\documentclass[a4paper]{jpconf}

\bibliographystyle{iopart-num}
\usepackage{citesort}
\usepackage{graphicx}
\usepackage{caption}
%\usepackage{subcaption}

\def\tilde{\raise.17ex\hbox{$\scriptstyle\sim$}}

\begin{document}
\title{Unlocking data: federated identity with LSDMA and dCache}

\author{AP Millar$^1$, G~Behrmann$^2$, C~Bernardt$^1$, P~Fuhrmann$^1$,
  M~Hardt$^3$, A~Hayrapetyan$^3$, D~Litvintsev$^4$, T~Mkrtchyan$^1$,
  A~Rossi$^4$, K~Schwank$^1$}

\address{$^1$ IT Dept., DESY, Notkestrasse 85, Hamburg, Germany}
\address{$^2$ Gerd Behrmann, Copenhagen, Denmark}
\address{$^3$ Steinbuch Centre for Computing, Hermann-von-Helmholtz-Platz 1, Karlsruhe, Germany}
\address{$^4$ Fermilab, Batavia, IL, USA}

\ead{paul.millar@desy.de}

\begin{abstract}
X.509, the dominant identity system from grid computing, has proved
unpopular for many user communities. More popular alternatives
generally assume the user is interacting via their web-browser. Such
alternatives allow a user to authenticate with many services with the
same credentials (user-name and password). They also allow users from
different organisations form collaborations quickly and simply.

Scientists generally require that their custom analysis software has
direct access to the data. Such direct access is not currently
supported by alternatives to X.509, as they require the use of a
web-browser.

Various approaches to solve this issue are being investigated as part
of the Large Scale Data Management and Analysis (LSDMA) project, a
German funded national R\&D project. These involve dynamic credential
translation (creating an X.509 credential) to allow backwards
compatibility in addition to direct SAML- and OpenID Connect-based
authentication.

We present a summary of the current state of art and the current
status of the federated identity work funded by the LSDMA project
along with the future road map.
\end{abstract}

\section{Introduction}
With the advent of the Large Hadron Collider (LHC) facility at CERN
and the corresponding world-wide grid infrastructure, the World-wide
LHC Computational Grid (WLCG), deploying and managing data that is
spread across the globe became a reality.  The community of scientists
that need to access this data is equally distributed, requiring a
common, decentralised authentication mechanism.

The adopted solution was X.509-based authentication.  Users are issued
with an X.509 certificate by their local certificate authority (CA),
which typically operate nationally.  The sites in WLCG that provide
resources trust CAs through their membership of a global trust
federation: the IGTF.  By trusting these CAs, services can trust that
the X.509 certificates they issue correctly identify the users.

Group-membership is an example of a service that authenticates via
X.509 certificate.  The user contacts the service, typically once per
day, to receive a signed assertion (in the form of an additional X.509
certificate) which describes in which groups the user is a member.
Later, when authenticating with other grid services, the user provides
both their identity and their group-membership assertion to the
service.  This allows the service to authorise the user based on their
group-membership, rather than their individual identity.

The adoption of X.509 client certificates led to a world-wide,
decentralised network of resources for the analysis and storage of LHC
data.  This model has spread, as the technology brought into
production to support LHC use-cases is adapted to support other
communities.  However, X.509 client certificates remain a burden for
users: it is neither a technology they are used to, nor something that
existing software provides a good user experience.  The extent to
which this affects scientists is not to be understated: it is only the
larger user communities that have adopted X.509 client certificates as
their authentication mechanism as they have the resources to train
their users in a new authentication mechanism.  The result is that
smaller communities are unable to access such resources.

Given the difficulties faced with X.509-based authentication, several
projects have looked into providing a decentralised (or ``federated'')
authentication mechanisms as alternatives.  Security Assertion Markup
Language (SAML) is perhaps the most widely deployed in academic
environments.  SAML allows the service to delegate responsibility for
authenticating the user to another service.  When the user
successfully authenticates, the service receives an assertion
describing that user.

\section{Trust}

Part of the problem faced by new users is in acquiring an X.509
certificate.  In part, this is because membership of the IGTF requires
the CA vets the user's identity with a face-to-face meeting and a
state-issued identity card or passport.  For new communities, this
process is hard as they often have no access to an IGTF-accredited CA
and establishing a new CA is prohibitively expensive.

This overhead is often unnecessary as the user's home institute
(university or research facility) may have already vetted the user's
identity with the level of scrutiny required by IGTF.  Several
services exist\cite{cilogon}\cite{tcs}\cite{dfn-aai} that allow a user
to acquire an X.509 user certificate in their web-browser based on
this scrutiny.  Typically SAML is used to allow the user to
authenticate with their home institute and the service will accept the
SAML assertion and provide the user with an X.509 certificate.  Such
services can provide the user with either provide long-lived (MICS
profile) or short-lived (SLCS profile) certificates, depending on the
intended use.

While this is an improvement, it carries two disadvantages: first, the
use of SAML currently requires the use of a web-browser; second, the
user is exposed to X.509 certificates, which generally require the
user to gain experience using specialist software.

The requirement to use a web-browser comes from SAML and how it is
commonly deployed.  Currently, the authentication service run by the
home institute and known as the identity provider (IdP), only support
the user contacting it with a web-browser.  The authentication
challenge is an HTML web-page, typically requesting the user's
user-name and password.  Such a web-page is free-form and there is no
practical way to automate this process.  An alternative SAML-based
authentication mechanism, called Enhanced Client or Proxy (ECP)
profile, allows for an automated login; however, there are currently
very few IdPs that support this profile, making it difficult to base a
service on ECP.  Without IdPs that support the ECP profile, it is
difficult to use SAML outside of the web-browser.

\section{Portals and delegation}

One solution is to provide the users with a web-based front-end
(portal) to mediate their computation and storage interactions.  Such
a portal acts as a graphical user interface (GUI) with which the user
can steer their analysis and data-management activity.  Once the user
authenticates with the portal using SAML Web-SSO, the portal uses the
SAML assertion to obtain an X.509 credential for that user.  It then
uses that credential when authenticating and interacting with grid
services on behalf of that user.  This has the benefit of ``hiding''
the X.509 credential from the user, while still allowing that user to
interact with grid services.

The other benefit from portal delegation is that all grid services
that require X.509 based authentication may be used by someone who
authenticates via SAML.  There is no requirement for services to be
updated.  This is necessary as upgrading the hundred of sites that
provide resources to WLCG is unfeasible.

To achieve this, the portal must obtain an X.509 credential
representing the user.  It is unpractical for the portal to run its
own CA as setting up a CA, running it, and gaining IGTF accreditation
is prohibitively expensive.  Therefore, it useful to acquire a
credential from some existing CA that supports the SLCS profile and
which is already accredited within the IGTF.  In Germany, the DFN
already provides such a service, but it is targeted at providing
certificates for users with web-browsers.  A custom API is available
for portal delegation, but this is not supported by existing portals.
Therefore, an protocol translation service is needed between DFN-AAI
and the existing portals.

\section{Non-browser based approaches}

When handling a much smaller set of resources, the restriction to
continue using X.509 credentials may be lifted.  For example, it is
feasible for a site to update the service it provides to accept direct
SAML-based authentication.  If the user's software can obtain a SAML
assertion then such an approach would allow the user to present this
SAML assertion when authenticating.

A web service may be provided that allows a user to download the SAML
assertion.  If this is stored in a standard location then client
software can locate and use it when authenticating.  While not ideal,
this provides a functional work-around to IdPs that support only
Web-SSO and not ECP.

With the SAML assertion saved in the user's local machine, it becomes
feasible to send the assertion as part of the login process.  Software
has been developed that can accept the SAML assertion as part of an
LDAP login process.  Since most server software can authenticate
against an LDAP service, this allows users to authenticate to these
services using their SAML assertion, provided the client software
accepts the SAML assertion as the password.

Another non-browser SAML approach is to use ECP and a SLCS profile CA.
In this approach, both the SLCS profile CA and the user's IdP support
ECP.  The client uses ECP to authenticate to the CA, which provides
the client with a short-lived X.509 certificate.  With this X.509
certificate, the client can interact with grid resources using the
X.509 credential.

While a wide-spread deployment of this solution is unfeasible due to
the lack of IdPs providing ECP support, for targeted groups with good
contact with their IdP, this becomes feasible.


\section{The work within LSDMA}

LSDMA\cite{lsdma} (Large Scale Data Management and Analysis) is a
project funded by the German government to link research of the
Helmholtz Association of research centres in Germany with community
specific Data Life Cycle Laboratories (DLCL). The DLCLs work in close
cooperation with scientists and they process, manage and analyse data
during its whole life cycle. The joint research and development
activities in the DLCLs lead to community-specific tools and
mechanisms. The DLCLs are complemented with a Data Services
Integration Team (DSIT). This provides generic technologies and
infrastructures for multi-community use, based on research and
development in the areas of data management, data access and security,
storage technologies and data preservation.

The dCache team at DESY and the group at KIT, as major partners within
the LSDMA DSIT team, are working on solving these authentication
problems through authenticating via SAML.

Within DSIT, the plan is to establish a federation of Identity
Providers (IdPs) for Germany: LSDMA-AAI.  This federation is similar
to the existing DFN federation (DFN-AAI)\cite{dfn-aai-website} and we
anticipate LSDMA-AAI becoming part of DFN-AAI in the future.  The
initial work will be on providing web-portals that allow a user to
authenticate via SAML.  An X.509 credential is built from that
information, which is used when authenticating with data servers.
This follows existing work in this
direction\cite{groeper2007architecture}.

Group-membership is a common concern across many projects as often it
is desired to authorise some operation based on that user's group
membership.  A common group-membership service will be run within
Germany to allow services to discover group membership of a user.

Current work involves establishing a credential-delegation service
that allows a portal to request an X.509 credential and the DFN-AAI
will mint an X.509 credential based on the user's SAML assertion, in
turn provided by Web-SSO.  This will enable portal delegation to work
with IdPs in Germany.

In parallel, work is underway in providing a service that allows users
to download their SAML assertion.  When a user then saves their
assertion, client software can use this when authenticating.

Yet another activity involves building clients that make use of ECP
and coordinating effort in enabling ECP in both IdPs and the DFN-AAI
SLCS CA service.

The final area of work is in enabling direct SAML-based
authentication.  This is either by embedding the assertion in the
password field with the LDAP facade, or by updating support in the
transfer protocols (e.g., FTP and NFS) to allow SAML-based
authentication.

The ultimate goal is to allow scientists to access their data and to
log into compute resources based solely on their home institute's
assertion of their identity.

\section{Conclusion}

We have discussed some of the issues surrounding authentication; in
particular, how the existing X.509-based grid infrastructure may be
used by a user who authenticates via SAML Web-SSO.  We also discussed
ways in which infrastructure may be updated to support direct
SAML-based authentication.

While considerable work remains, there are considerable benefits from
allowing users to authenticate via their home institute.  Reducing
barriers and making it easier to access computing and storage
resources will empower scientists in many disciplines to maximise the
scientific output from the available data by maximising the utility of
available resources.

\ack

Work described in this paper was funded by the LSDMA project, DESY,
Fermilab and NDGF.

\section*{References}
\bibliography{LSDMA-CHEP2015}

\end{document}
