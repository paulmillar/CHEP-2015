\documentclass[a4paper]{jpconf}

\bibliographystyle{iopart-num}
\usepackage{citesort}
\usepackage{graphicx}
\usepackage{caption}
%\usepackage{subcaption}

\def\tilde{\raise.17ex\hbox{$\scriptstyle\sim$}}

\begin{document}
\title{Unlocking data: federated identity with LSDMA and dCache}

\author{AP Millar$^1$, G~Behrmann$^2$, C~Bernardt$^1$, P~Fuhrmann$^1$,
  M~Hardt$^3$, A~Hayrapetyan$^3$, D~Litvintsev$^4$, T~Mkrtchyan$^1$,
  A~Rossi$^4$, K~Schwank$^1$}

\address{$^1$ IT Dept., DESY, Notkestrasse 85, Hamburg, Germany}
\address{$^2$ Gerd Behrmann, Copenhagen, Denmark}
\address{$^3$ Steinbuch Centre for Computing, Hermann-von-Helmholtz-Platz 1, Karlsruhe, Germany}
\address{$^4$ Fermilab, Batavia, IL, USA}

\ead{paul.millar@desy.de}

\begin{abstract}
X.509, the dominant identity system from grid computing, has proved
unpopular for many user communties. More popular alternatives
generally assume the user is interacting via their web-browser. Such
alternatives allow a user to authenticate with many services with the
same credentials (username and password). They also allow users from
different organisations form collaborations quickly and simply.

Scientists generally require that their custom analysis software has
direct access to the data. Such direct access is not currently
supported by alternatives to X.509, as they require the use of a
web-browser.

Various approaches to solve this issue are being investigated as part
of the Large Scale Data Management and Analysis (LSDMA) project, a
German funded national R\&D project. These involve dynamic credential
translation (creating an X.509 credential) to allow backwards
compatibility in addition to direct SAML- and OpenID Connect-based
authentication.

We present a summary of the current state of art and the current
status of the federated identity work funded by the LSDMA project
along with the future road map.
\end{abstract}

\section{Introduction}
With the advent of the Large Hadron Collider (LHC) facility at CERN
and the corresponding world-wide grid infrastructure, the World-wide
LHC Computational Grid (WLCG), deploying and managing data that is
spread across the globe became a reality.  The community of scientists
that need to access this data is equally distributed, requiring a
common, decentralised authentication mechanism.

The solution was to adopt X.509-based authentication.  Users are
issued with an X.509 certificate by their local certificate authority
(CA), which typically operate nationally.  The sites in WLCG that
provide resources trust CAs through their membership of a global trust
federation: the IGTF.  By trusting these CAs, services can trust that
the X.509 certificates they issue correctly identify the users.

Group-membership is an example of a service that authenticates via
X.509 certificate.  The user contacts the service, typically once per
day, to receive a signed assertion (in the form of an X.509
certificate) which describes in which groups the user is a member.
Later, when authenticating with other grid services, the user provides
both their identity and their group-membership assertion to the
service.  This allows the service to authorise the user based on their
group-membership, rather than their individual identity.

The adoption of X.509 client certificates led to a world-wide,
decentralised network of resources for the analysis and storage of LHC
data.  This model has spread, as the technology brought into
production to support LHC use-cases is adapted to support other
communties.  However, X.509 client certificates remain a burden for
users: it is neither a technology they are used to, nor something that
existing software provides a good user experience.  The extent to
which this affects scientists is not to be understated: it is only the
larger user communites that have adopted X.509 client certificates as
their authentication mechanism as they have the resources to train
their users in a new authentication mechanism.

Given the difficulties faced with X.509-based authentication, several
projects have looked into providing a decentralised (or ``federated'')
authentication mechanisms as alternatives.  Security Assertion Markup
Language (SAML) is perhaps the most widely deployed in accademic
environments.  SAML allows the service to delegate responsibility for
authenticating the user to another service.  When the user
successfully authenticates, the service receives an assertion
describing that user.

\section{Trust}

Users new to grid computing often experience problems in acquiring an
X.509 certificate.  In part, this is because membership of the IGTF
requires the CA vet the users identity with face-to-face meeting and
state-issued identity card.  For new communities, this process is hard
as they often have no access to an IGTF-accredited CA and establishing
a new CA is prohibitively expensive.

This overhead is often unnecessary as the user's home institute may
have already vetted the user's identity with the level of scrutiny
required by IGTF.  Several projects exist that allow a user to acquire
an X.509 certificate in their web-browser.  Typically SAML is used to
allow the user to authenticate with their home institute and the
service will accept the SAML assertion and provide the user with an
X.509 certificate.  Such services can provide the user with either
provide long-lived (MICS profile) or short-lived (SLCS profile)
certificates, depending on the intended use.

While this is an improvement, it carries two disadvantages: first, the
use of SAML currently requires the use of a web-browser; second, the
user is exposed to X.509 certificates, which generally require the
user to gain experience using specialist software.

\section{Direct access problem}

The requirement to use a web-browser comes from SAML.  Currently, the
authentication services, known as identity providers (IdP), only
support the user contacting it with a web-browser.  The authentication
challenge is an HTML web-page, typically requesting the user's
username and password.  Such a web-page is free-form and there is no
pratical way to automate this process.  An alternative SAML-based
authentication mechanism, called Enhanced Client or Proxy (ECP)
profile, allows automated login; however, there are currently very few
IdPs that support this profile, making it difficult to base a service
on ECP.

\section{Portal delegation problem}

Portals provide considerable benefits to ...

SAML web-SSO may be used.

Needs a credential translation service.

Several approaches already tried: SERONGS, InCommons.

Services that have already enabled support for MyProxy-like credential
delegation.

\section{LSDMA}

LSDMA\cite{lsdma} (Large Scale Data Management and Analysis) is a
project funded by the German government to link research of the
Helmholtz Association of research centres in Germany with community
specific Data Life Cycle Laboratories (DLCL). The DLCLs work in close
cooperation with scientists and they process, manage and analyse data
during its whole life cycle. The joint research and development
activities in the DLCLs lead to community-specific tools and
mechanisms. The DLCLs are complemented with a Data Services
Integration Team (DSIT). This provides generic technologies and
infrastructures for multi-community use, based on research and
development in the areas of data management, data access and security,
storage technologies and data preservation.

dCache, as major partner within the LSDMA DSIT team, is working on
solving these authentication problems through authenticating via SAML.
This is non-trivial as almost all SAML usage has been for web-driven
activity while most data transfers are not initiated by a web-browser.

Within DSIT, the plan is to establish a federation of Identity
Providers (IdPs) for Germany: LSDMA-AAI.  This federation is similar
to the existing DFN federation (DFN-AAI)\cite{dfn-aai-website} and we
anticipate LSDMA-AAI becoming part of DFN-AAI in the future.  The
initial work will be on providing web-portals that allow a user to
authenticate via SAML.  An X.509 credential is built from that
information, which is used when authenticating with data servers.
This follows existing work in this
direction\cite{groeper2007architecture}.

Group-membership is a common concern across many projects as often it
is desired to authorise some operation based on that user's group
membership.  A common group-membership service will be run within
Germany to allow services to discover group membership of a user.

Later, support in dCache for certain transfer protocols (e.g., FTP and
NFS) will be updated to allow direct SAML-based authentication.  This
will allow clients to access their data directly without a web-portal
using their home institute credentials in a secure fashion.


\section{Work undertaken}

LDAP facade provides the ability to add SAML interop with existing
software.

Tunnelling SAML assertion through ssh password field.

Agreement with DFN-AAI to use their SLCS service.

\section{Future work}

Deployment of a credential translation service to bridge between an
OAuth-based X.509 request

convert SAML assertions to X.509 

\section{Conclusion}

\ack

Work described in this paper was funded by the LSDMA project, DESY,
Fermilab and NDGF.

\section*{References}
\bibliography{LSDMA-CHEP2015}

\end{document}
