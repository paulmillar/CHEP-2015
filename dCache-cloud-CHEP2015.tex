\documentclass[a4paper]{jpconf}

\bibliographystyle{iopart-num}
\usepackage{citesort}
\usepackage{graphicx}
\usepackage{caption}
%\usepackage{subcaption}

\def\tilde{\raise.17ex\hbox{$\scriptstyle\sim$}}

\begin{document}
\title{dCache, Sync-and-Share for Big Data}

\author{P~Fuhrmann$^1$, AP Millar$^1$, T~Mkrtchyan$^1$,
  G~Behrmann$^2$, C~Bernardt$^1$, K~Schwank$^1$, A~Rossi$^3$,
  D~Litvintsev$^3$, P~van~der~Reest$^1$, V~Guelzow$^1$,
  Q~Buchholz$^1$}

\address{$^1$ IT Dept., DESY, Notkestrasse 85, Hamburg, Germany}
\address{$^2$ NORDUnet, Copenhagen, Denmark}
\address{$^3$ Fermilab, Chicago, IL, USA}

\ead{paul.millar@desy.de}

\begin{abstract}
The availability of cheap, easy-to-use sync-and-share cloud services
has split the scientific storage world into the traditional big data
management systems and the very attractive sync-and-share
services. With the former, the location of data is well understood
while the latter is mostly operated in the Cloud, resulting in a
rather complex legal situation.

Beside legal issues, those two worlds have little overlap in user
authentication and access protocols. While traditional storage
technologies, popular in HEP, are based on X509, cloud services and
sync-n-share software technologies are generally based on
user/password authentication or mechanisms like SAML or Open ID
Connect. Similarly, data access models offered by both are somewhat
different, with sync-n-share services often using proprietary
protocols.

As both approaches are very attractive, dCache.org developed a hybrid
system, providing the best of both worlds. To avoid reinvent the
wheel, dCache.org decided to embed another Open Source project:
OwnCloud. This offers the required modern access capabilities but does
not support the managed data functionality needed for large capacity
data storage.

With this hybrid system, scientist can share files and synchronize
their data with laptops or mobile devices as easy as with any other
cloud storage service. On top of this, the same data can be accessed
via established mechanisms, like GridFTP to serve the Globus Transfer
Service or the WLCG FTS3 tool, or the data can be made available to
worker nodes or HPC applications via a mounted filesystem. As dCache
provides a flexible authentication module, the same user can access
its storage via different authentication mechanisms; e.g., X.509 and
SAML. Additionally, users can specify the desired quality of service
or trigger media transitions as necessary, so tuning data access
latency to the planned access profile. Such features are a natural
consequence of using dCache.

We will describe the design of the hybrid dCache/OwnCloud system,
report on several months of operations experience running it at DESY,
and elucidate on the future road-map.
\end{abstract}

\section{Introduction}

This\cite{rfc3230}


\section{Conclusion}

\ack

Acknowledge LSDMA

\section*{References}
\bibliography{dCache-cloud-CHEP2015}

\end{document}
