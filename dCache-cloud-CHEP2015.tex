\documentclass[a4paper]{jpconf}

\bibliographystyle{iopart-num}
\usepackage{citesort}
\usepackage{graphicx}
\usepackage{caption}
%\usepackage{subcaption}

\def\tilde{\raise.17ex\hbox{$\scriptstyle\sim$}}

\begin{document}
\title{dCache, Sync-and-Share for Big Data}

\author{AP Millar$^1$, P~Fuhrmann$^1$, T~Mkrtchyan$^1$,
  G~Behrmann$^2$, C~Bernardt$^1$, K~Schwank$^1$, A~Rossi$^3$,
  D~Litvintsev$^3$, P~van~der~Reest$^1$, V~Guelzow$^1$,
  Q~Buchholz$^1$}

\address{$^1$ IT Dept., DESY, Notkestrasse 85, Hamburg, Germany}
\address{$^2$ Gerd Behrmann, Copenhagen, Denmark}
\address{$^3$ Fermilab, Chicago, IL, USA}

\ead{paul.millar@desy.de}

\begin{abstract}
The availability of cheap, easy-to-use sync-and-share cloud services
has split the scientific storage world into the traditional big data
management systems and the very attractive sync-and-share
services. With the former, the location of data is well understood
while the latter is mostly operated in the Cloud, resulting in a
rather complex legal situation.

Beside legal issues, those two worlds have little overlap in user
authentication and access protocols. While traditional storage
technologies, popular in HEP, are based on X.509, cloud services and
sync-and-share software technologies are generally based on
user/password authentication or mechanisms like SAML or Open ID
Connect. Similarly, data access models offered by both are somewhat
different, with sync-and-share services often using proprietary
protocols.

As both approaches are very attractive, dCache.org developed a hybrid
system, providing the best of both worlds. To avoid reinvent the
wheel, dCache.org decided to embed another Open Source project:
OwnCloud. This offers the required modern access capabilities but does
not support the managed data functionality needed for large capacity
data storage.

With this hybrid system, scientist can share files and synchronize
their data with laptops or mobile devices as easy as with any other
cloud storage service. On top of this, the same data can be accessed
via established mechanisms, like GridFTP to serve the Globus Transfer
Service or the WLCG FTS3 tool, or the data can be made available to
worker nodes or HPC applications via a mounted filesystem. As dCache
provides a flexible authentication module, the same user can access
its storage via different authentication mechanisms; e.g., X.509 and
SAML. Additionally, users can specify the desired quality of service
or trigger media transitions as necessary, so tuning data access
latency to the planned access profile. Such features are a natural
consequence of using dCache.

We will describe the design of the hybrid dCache/OwnCloud system,
report on several months of operations experience running it at DESY,
and elucidate on the future road-map.
\end{abstract}

\section{Introduction}

The term ``cloud storage'' has come to have different meanings to
different people; for example, a large file-system, a generic
object-storage and a sync-and-share style service have all been deemed
cloud storage.  Perhaps the most common aspect is that a large
(seemingly unlimited to the users) remote storage service is provided,
often with a direct-charge business model that tries to ensure the
sustainability of that service.

Storing large amounts of data is not a new concept to the high-energy
particle physics (HEP) community, which has constantly strived against
hardware limitations to store data.  The result is software, such as
dCache\cite{dcache}, that combines the storage capacity of many
computers, allowing sites to offer a scalable storage service both in
terms of performance and capacity.  This allows sites to store both
the HEP data produced by detectors and the result of physics analysis.

Despite this experience with large amounts of data, users found the
services offered by sync-and-share cloud companies a useful addition
to the services offered by sites.  Synchronising files between
desktop(s) and laptop(s), and easily sharing those files with
colleagues provided a convenient and light-weight solution compared to
more heavy-weight solutions that were officially supported by sites.

Allowing disconnect synchronision (which does not require pairs of
devices be available concurrently) means that files must be stored on
some central service, which may be located in a different country from
the user.  Within a changing climate, it became desirable not to
expose such files to external companies, especially those based in
foreign countries.

For these reasons, DESY needed to provide a cloud sync-and-share
service for its users: most promenantly, to particle physicists and
photon scientists.  This must provide a similar level of service, with
easy replication between services, but with data stored on-site.

In this paper, we present this sync-and-share service, its design and
implementation, the current status and planned future work.

\section{Requirements}

The most basic requirement of the DESY sync-and-share service derives
from DESY users' prior experience with external services.  It must be
``easy to use'' or sufficiently similar to existing services that
users can migrate to the new system with little or no support.

\begin{itemize}
\item
It should support all the computing platforms DESY currently employs:
Microsoft Windows based desktops, Apple Macintosh and Linux-based
computers.  There should also be support for mobile devices, such as
those that have adopted the Android Operating System and iOS.

\item
All data must be stored at DESY, so users can be confident that their
data has not been exported to third-parties for data-mining or other
unauthorised activity.

\item
The service should integrate well with the existing DESY
infrastructure; in particular, the user should be able to authenticate
with their DESY username and password.
\end{itemize}

Although not hard requirements, the following describes some
additional use-cases that are anticipated.  These are not required for
the service to be useful, but are likely to be requested by users.

\begin{itemize}
\item
The users may want some storage capacity that is not synchronised with
their devices.  Such files will be available for sharing with others
without being synchronised on all devices.  Files should be movable
between synchronising and non-synchronising space.

\item
Providing different quality-of-service guarantees: archival storage
for mission critical data and disk-only copy for other data.

\item
Supporting transferring of files to some remote service without the
data travelling through the user's computer.

\item
 Direct access to storage files from computing facilities.
\end{itemize}

\section{Design}

The project evaluated different existing solutions and, while no
single project satisfied all requirements, a combination of two
projects satisfied the existing requirements and provided sufficient
flexibility to satisfy the anticipated additional usage.  These two
projects are ownCloud\cite{owncloud-website} and dCache.

The ownCloud project is an open-source sync-and-share storage solution
that is, at the time of writing, gaining popularity.  In particular,
it is the use of this service in other HEP laboratories that makes
ownCloud particularly attractive.  Although ownCloud provides users
with a good front-end system, including web-based browsing, and
synchronisation clients for Microsoft Windows, Linux-based OSes, Mac
OS X and mobile devices, it lacks support for managing data on large
scale.  Instead, it assumes a site has an appropriate storage system
mounted on the local filesystem.

The dCache project provides a powerful managed storage system, with
good integration with scientific data life-cycle and automatic file
migration between different media.  It currently provides no
sync-and-share facilities.

The DESY Cloud design combines both dCache and ownCloud.  dCache's
support for standard protocols allows for easy integration with
ownCloud; dCache is mounted via NFS v4.1\cite{rfc5661} on the RHEL 7
servers hosting the ownCloud software.  The ownCloud instances access
files using the standard POSIX API.

In ownCloud, user authentication is based on username and password.
The user-supplied credential is used to authenticate an LDAP query,
where the LDAP server uses Kerberos interaction to verify the supplied
credential (a ``pass through'' configuration).  The LDAP query
provides information about the user, which is used by ownCloud to
establish whether or not the user is allowed to use the service.

At DESY, a proprietary system (called ``the registry'') provides
information about users.  Various adapters (called ``plaform
adapters'') allow custom integration of this information with the
various services DESY provide.  For the DESY Cloud service, there are
two parts to the platform adapter: the registry platform adapter and
the server-side platform adapter.

The registry platform adapter is located as part of the registry
service.  It provides information for the LDAP server and sends events
to the server-side platform adapter when an account changes state.
Such changes are part of that account's life-cycle.  Events include
when an account is created, when it expires, when it is either enabled
or disabled, and when an account should be archived.

The server-side platform adapter updates dCache in response to events
from the registry platform adapter; for example, when it receives a
account-created event, a suitable home directory is created for that
user.  Sufficient information is included in the event for the
server-side platform adapter to operate correctly.

Several ownCloud instances are run concurrently on different machines.
These form the individual services in a load-balanced pool, with
client requests being directed to one machine.  In addition to scaling
the front-end service, this load-balancing allows routine maintenance
of the ownCloud machines without service down-time.

\section{Scientific Cloud Vision}

\begin{figure}
  \centering
  \includegraphics[width=\textwidth]{Figures/dCache-cloud}
  \caption{The Scientific Cloud}
  \label{fig:scicloud}
\end{figure}

The DESY Cloud service fits within the dCache ``Scientific Cloud''
vision.  In this model, a single storage system provides storage
capacity that may be used in different operational modes.  For
example, the same storage system might be used for fast ingest of
data, a sync-and-share service, storage for HPC and HTC computing
farms, and remote storage for moving data to remote sites.
Figure~\ref{fig:scicloud} shows this diagrammatically.

Providing a single system brings many advantages.  A single system
reduces the cost of operating the system as different services appear
as different facets of the same underlying storage.  The tight
integration can provide a better end-user experience as there is no
scripts copying files between systems.  In can lead to better use of
resources, as unnecessary redundances can be removed.

\section{Current status}

The initial prototyping and deployment has been completed.  A
production infrastructure has been developed and is being rolled out
within the DESY community.

At time of writing, the current set of users is small: only some
200~users have access to the service.  They currently have around
2~million files stored on the service, consuming some 2.4~TiB of
capacity.

The experience gained from this limited deployment has allowed the
deployment team to discover problems both in ownCloud and dCache
software.  The team at DESY has been proactive in fixing these
problems, with patches being sent.  These patches have been accepted
in new releases of both dCache and ownCloud; this is in keeping with
the desired goal of running unpatched software.

This experience, both in running the software and DESY's ability to
diagnose and fix problems as they are found, leave us confident as we
roll out the software to all members of DESY.

\section{Development and future work}

One future goal is to allow direct access to the files a user has
stored via DESY Cloud service.  Such access would allow the user to
run analysis work based on analysis programs files kept in sync with
their work machines.  It would also allow users to synchronise the
output from their analysis work and to share those results with their
colleagues.

To support this usage, the files written into dCache must have the
users ownership.  Normally, files written by an application have that
application's ownership; for example, all files written by the Apache
daemon (that hosts the ownCloud service) are normally owned by user
{\em httpd}, rather than the DESY user.  It proved too difficult to
fix this within ownCloud, so a simple work-around was introduced in
dCache.

Another common use-case involves the bulk transfer of files between
two sites, perhaps in different countries.  Services
exist\cite{fts-website}\cite{globus-website} that support transferring
scientific data reliably, optimising throughput and retrying transfers
as necessary.  By allowing access through the protocols used in such
services, the DESY Cloud service would allow users to migrate data as
necessary.

dCache already allows files to be shared between users by specifying
ACLs that permit an individual to read that file.  Currently these are
decoupled from the ownCloud sharing database: modifying a file's ACL
in dCache has no effect on whether the file is shared in ownCloud;
likewise, sharing or unsharing a file in ownCloud has no effect on
dCache's ACLs.  Future work will link these two, allowing consistent
permissions whether the file is read through ownCloud or directly via
dCache.  To facilicate this, ownCloud and dCache use a common database
technology: PostgreSQL.

Another issue is that of data integrity and preventing data
corruption.  The concept of preserving data integrety by computing a
checksum is baked into dCache, with background checking (scrubbing)
and automatic checking on integrety when transfering files.  Such
concepts currently do not exist within ownCloud.  Working with the
ownCloud developers, we will introduce such checks to ensure
synchronisation is safe from corruption.

One design feature of dCache is that it will redirect the client to
read or write data directly to the computer that stores (or will
store) the request data.  Such redirection allows dCache to scale to
large storage capacity.  Currently ownCloud does not support
redirection and we will be working with ownCloud to add this support
into their sync client.

One final point is with file change notification.  Currently the
ownCloud client polls for changes.  Since most times the request
reveals that nothing has changed, this introduces unnecessary load on
the server.  To reduce the impact of this load, the client checks
whether there have been any changes every 30 seconds.  However, this
results in the client becoming slow to react to remote changes.

Introducing a notification scheme would allow clients to react quickly
while not placing to great a burden on the server.

\section{Conclusion}

We have presented the DESY Cloud service: a sync-and-share service in
production at DESY using open-source software.  While the project is
still being rolled out and currently has a limited number of active
users, our experience gives us confidence it will scale to support all
DESY users.

The improvements and bug-fixes that were developed while bringing this
service into a production-like state and the goal of running unpatched
software results in a service that other sites can replicate.  For
example, sites that already have a dCache instance can extend that
service to include a sync-and-share service.

\ack

Work described in this paper was funded by the LSDMA project, DESY,
Fermilab and NDGF.

\section*{References}
\bibliography{dCache-cloud-CHEP2015}

\end{document}
