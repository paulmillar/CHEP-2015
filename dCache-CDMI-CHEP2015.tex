\documentclass[a4paper]{jpconf}

\bibliographystyle{iopart-num}
\usepackage{citesort}
\usepackage{graphicx}
\usepackage{caption}
%\usepackage{subcaption}

\def\tilde{\raise.17ex\hbox{$\scriptstyle\sim$}}

\begin{document}
\title{New adventures in storage: cloud storage and CDMI}

\author{AP Millar$^1$, G~Behrmann$^2$, C~Bernardt$^1$, P~Fuhrmann$^1$,
  D~Litvintsev$^4$, T~Mkrtchyan$^1$, A~Rossi$^4$, K~Schwank$^1$,
  H~Heßling$^3$, J~Weschenfelder$^3$}

\address{$^1$ IT Dept., DESY, Notkestrasse 85, Hamburg, Germany}
\address{$^2$ NORDUnet, Copenhagen, Denmark}
\address{$^3$ HTW Berlin}
\address{$^4$ Fermilab, Chicago, IL, USA}

\ead{paul.millar@desy.de}

\begin{abstract}
Traditionally storage systems have had well understood
responsibilities and behaviour, codified by the POSIX standards. More
sophisticated systems (such as dCache) support additional
functionality, such as storing data on media with different latencies
(SSDs, HDDs, tapes). From a user's perspective, this forms a
relatively simple adjunct to POSIX: providing optional
quality-of-service values when writing data and optionally requesting
data be staged from tape ahead of use.

The CDMI protocol provides a standard mechanism for clients to
discover and use many advanced features. Such features include storing
and querying metadata, searching for files matching metadata
predicates, controlling a file's quality-of-service and retention
policies, providing an object store and alternative protocol
discovery.

A CDMI enabled storage has the potential for greatly simplifying a
more general service as some high-level functionality can be delegated
to the storage system. This reduces and may remove the need to run
additional services, which makes it easier for sites to support their
users.

By implementing the CDMI standard, dCache can expose new features in a
standards compliant fashion. Here, various scenarios are presented
where CDMI provides an advantage and the road-map for CDMI support in
dCache is explored.
\end{abstract}

\section{Introduction}

This\cite{rfc3230}

\section{Conclusion}

\ack

Acknowledge LSDMA
Acknowledge HTW Berlin

\section*{References}
\bibliography{dCache-CDMI-CHEP2015}

\end{document}
